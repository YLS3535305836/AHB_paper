%随着全球科技的迅速发展,电子设备的智能化程度随着集成电路产生的发展也越来越高,在日常生活在,开关电源作为各种电子产品的核心模块也愈发的被社会所重视。随着芯片的发展,各种软件层出不穷,但电池材料的发展不够迅速,大众对充电器的功率和充电速度的要求日益提高。反激式AC/DC开关电源变换器由于其隔离性能优异、设计简单和成本较低等因素活跃于中低功率的快充领域。但由于其损耗较大和开关频率较低的特性已逐渐跟不上电子设备的发展,为了满足社会日益旺盛的需求,探索更多的反激式变换器的可能性,本文设计了一款基于非对称半桥拓扑结构的反激式开关电源变换器芯片用于中等功率环境的快充领域。

%本文设计的基于副边反馈结构的非对称半桥反激式变换器芯片,通过多种工作模式在全负载范围内都可以实现功率管的零电压导通,极大地降低功率管的开关损耗,提高电路的转换效率。采用最适配非对称半桥反激式变换器的谐振谷值开关(RVS)模式,在轻载工况下同时实现高低边功率管的零电压导通,新颖性的设计了精确谷底导通模块在谐振谷最低点导通低边功率管,实现最小的开关损耗的同时降低采样电压的导通过冲问题。还设计了退磁时间动态校准模块,实现了原边电感最佳能量的传递并完成了副边二极管零电流关断降低导通损耗,进一步提高电路的转换效率。此外芯片还加入了谷值锁定电路防止出现跳谷问题引起的开关频率波动等不稳定问题,加入峰值电流控制电路实现宽范围输出下副边反馈电压的稳定,满足片内多种电路的控制要求;加入

