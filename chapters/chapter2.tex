\chapter{绪论}

\section{研究背景及意义}


自从全球信息化浪潮的到来,得益于各种电子产品的不断涌现,集成电路的发展愈发的得到社会的广泛重视。在各类集成电路芯片中,稳定的电源芯片更是重中之重,是一台电子设备可以正常运行的基石和保障。
自1958年第一块集成电路芯片诞生以来,电源芯片几十年的发展的过程中,逐渐被分为了线性电源和开关电源两大类。
这两类电源芯片逐渐发展出了不同的特点和工作领域,线性电源结构简单,通过线性操作转换电压产生高质量无纹波的稳定输出电压,且较大的带宽可以产生更快的瞬态响应,但其通过传输晶体管转换大电压的方式损失大量能量,转换效率较低,为了解决产生的大量热损耗加入的散热设计影响设备体积,故并不适合于小型便携式电子设备所要求的高效率低功耗等特点。
开关电源是通过对开关晶体管进行通断控制,实现对电感电容等储能元件进行周期性的充放电操作,最终产生稳定的输出电压和输出电流。
相比于线性电源,开关电源虽然有较大的输出纹波,但由于其体积小、成本低、效率高、以及便携性和可靠性等特点,被广泛利用于工农业生产、交通运输、航空航天、军事以及消费电子等各个领域,市场规模庞大\cite{shen2012,ic_power_management_book}。

开关电源以输入电压是直流电压或交流电压分为DC/DC开关电源和AC/DC开关电源两类,DC/DC开关电源的变换器芯片的制造工艺和相关的设计要求和方案都已经有了非常成熟的体系和相关的约束,众多的DC/DC变换器芯片收到了社会各界的认可。但相比于DC/DC开关电源,AC/DC开关电源变换器芯片由于其需要通过整流桥将市交流电整流滤波为高压直流电,芯片的制造需要采取相应的高压工艺,不仅设计过程中遇到各种困难,产品的可靠性和转换效率也难以提升。随着最近十几年来各类手机电脑等设备的智能化,其对更高功率、更小体积和更低功耗的AC/DC电源适配器的市场需求极其迫切,因此高开关频率、高功率密度、高转换效率以及智能化是AC/DC变换器的主要发展方向。

目前AC/DC开关电源变换器芯片主要采用的拓扑包括正激式(Forward)拓扑、反激式(Flyback)拓扑、谐振式(LLC)拓扑等。不同的拓扑都有其优缺点以及最佳的应用场合,其中LLC拓扑在原边采用谐振腔实现了软开关技术,效率更高,适合更大功率和高功率密度的应用场景,但其副边的全波整流结构并不适合宽范围输出,导致其应用产生了局限性,不适合中等功率电源适配器的使用\cite{hu2013,zhang2004};同时与正激式拓扑相比,反激式拓扑不仅控制策略较为简单,且只需更少的片外元器件,电路板面积更小,因此反激式拓扑目前是应用于便携式电源适配器的主流拓扑。但反激式拓扑由于其结构特性,不仅存在电压电流应力高的问题,而且由于其硬开关的原因,即使使用准谐振技术也难以实现全负载下功率管的零电压开通(ZVS),开关损耗较大,很难做到较大的转换效率以及高开关频率\cite{li2019analysis}。虽然很多论文\cite{spiazzi2011high,tarzamni2016full,huang2016design}都提出了有源钳位反激变换器,实现了软开关操作,但其较大的电压应力仍影响着外围器件尤其是变压器的选择,难以缩小电源适配器的体积。

为了解决传统式反激拓扑和有源钳位拓扑的诸多问题,非对称半桥反激式(AHBF)拓扑逐渐受到了更多的关注。直到现在,诸多论文和出版物对该拓扑进行了出色的分析\cite{chen2002analysis,kim2012analysis,huber2017analysis},与其他反激式拓扑相比,非对称半桥反激式拓扑不仅能够通过原边电感电流的续流特性在全负载范围内实现功率管的零电压导通,大大降低开关损耗,有更高的功率密度,其类似于LLC拓扑的原边结构更具有小电压应力的优点,在外围元器件选择和变压器体积上有了更大的余量,且不同于LLC拓扑,非对称半桥反激式拓扑的副边类似传统反激式拓扑,可以实现更宽的输出范围,允许非对称半桥反激式结构适用于中等功率的电源适配器中,非常符合现代社会对高功率低功耗的快充设备的需求。

\section{ACDC国内外研究进展}

根据上文叙述,非对称半桥反激式开关电源拓扑在各方面都有很强的竞争性,但其由于谐振特性在实现高开关频率、高功率密度和高转换效率方面仍有很多难点未被攻克。本节将结合国内外相关参考文献,从非对称半桥拓扑的发展历程和针对上述各种关键需求,对国内外包括但不限于非对称半桥反激式变换器控制方案研究进行详细调研分析总结。

非对称半桥反激式变换器的拓扑结构在2000年由D.H.Seo等人提出,通过严谨的计算验证了该拓扑的可行性,证明其克服了其他非对称PWM变换器未能将变压器漏感作为谐振元件和相较于有源钳位反激式变换器高电压应力的缺点,既实现了软开关技术又适合在高输入电压应用\cite{seo2000_AHB}。

2002年Chern-Lin Chen等人给出了非对称半桥反激式变换器的工作原理和数学分析,以降低开关损耗,并对输入直流电压为400V的电路进行了分析\cite{chen2002_AHB,chen2002_AHB1}。

2004年X.Xu等人为了实现电路功率管损耗的最小化,根据变压器漏感、功率管寄生电容、栅极驱动信号延迟和负载电流等关键参数对ZVS导通和ZCS关断进行了详细的理论分析和电路验证\cite{xu2004_AHB}。

2005年D.Y Huh等人为了实现更低的待机功耗提出了突发模式(Burst Mode,BM)控制的新技术,引用于空载和极轻载情况\cite{choi2005_BM1,lo2008_BM2}。
同年Bor-Ren Lin等人使用开关管替代续流二极管,在非对称半桥反激式变换器的输出端应用了同步整流技术,并进行了详细的分析和实验证明了该技术能够有效的降低导通损耗,提高转换效率\cite{lin2005_AHB}。

2006年Tso-Min Chen等人对非对称半桥反激式变换器进行了功率级的小信号建模,分析了储能元件和峰值电流模式控制对传输特性的影响,指出该小信号模型的传递函数是由两个低频极点和两个高频极点组成的四阶系统,如果采用峰值电流模进行控制则将降低到三个极点,包括一个由输出电容和负载确定的主极点和两个由谐振电感、谐振电容确定的高频极点\cite{chen2006_AHB}。




2011年浙江大学Junming Zhang等人提出利用迟滞延迟时间控制\cite{zhang2011_vallye_switch1},从而达到谷值锁定目的。

为了使得谷值锁定方案能够灵活运用于不同配置的变换器,并使得控制系统更加可靠,2014年韩国科学技术院Ju-Pyo Hong等人提出数字控制的谷值锁定方案如图1.5所示\cite{hong2014_vallye_switch2}。该方案利用峰值电流构建功率重叠范围,采用环路逼近的形式,将谷值增和谷值减的信号传输到寄存器,通过算法实现可靠的谷值逼近。

为解决空载下的静态损耗,防止采用高压取电为控制芯片供能引入的巨大损耗,同时采用辅助绕组供电也存在控制初始态问题。对此,SangCheol Moon等人通过使用有源器件阻断电流通过无源器件,从而减少高压输入端引入的损耗,也能够解决初始状态供电问题\cite{moon2011new_static_loss1}。


国立台湾大学Chia-Jung Chang等人通过重新设计副边稳压源来反转反馈电压,以实现轻载低损耗\cite{chang2012_static_loss2},光耦合器以及其对应的输入输出支路在轻负载条件下的功耗得到有效降低,控制芯片的轻载效率得到提升。

东南大学Qinsong Qian等人提出一种能够应用于原边反馈和副边反馈的高效数字控制方案\cite{qian2022high_ZCS1},具体方案如图1.11所示。该方案通过检测辅助绕组上的电压降,来判断SR功率管的ZCS关断情况。该控制方案在SR功率管实现ZCS关断后,关断谐振腔中的功率管,有效地降低了变换器导通损耗。
%Li-Ming Wu 和 Chen-Yin Pong 认为不对称半桥反激变换器中的谐振电感和隔直电容是通过谐振传递 能量,是能量传递的部件,而不仅仅是进行线性充放电的器件;Jee-Hoon Jung 和 Joong-Gi Kwon 设计了限制不对称半桥反激变换器的软开关,并分析了最佳谐振条 件,使其能够在大电流场合拥有很高的率效;G. - Y. J e o n g 在不对称半桥反激变换器 中采用了自驱动电压型的同步整流方法,并对其进行了分析;Han Li 对不对称 半桥反激变换器的高频工作状态进行了分析,得到了变换器的稳态电压传输方程和 电压传输函数,并证明了通过电压传递函数设计的变换器比通过线性模型设计的变 换器更易于控制;Sichirollo F.等将不对称半桥反激变换器应用到高亮发光二极管 (HLLED)镇流器中,其开关频率高达 450kHz。

不对称半桥反激变换器具有结构简单,效率高、成本低等优点,非常适合应用 于笔记本电脑的适配器、通信电源和打印机等中小型家用电器的电源,是一种非常 具有研究价值和应用前景的电路拓扑。

\section{论文结构安排}


论文章节安排如下: 

第一章绪论阐述了AC/DC反激式开关电源的背景和意义,分析比较了非对称半桥反激式变换器国内外的研究情况,最后对论文的章节内容进行安排。 

第二章具体描述了各种反激式变换器的拓扑结构和非对称半桥反激式变换器工作原理,结合关键信号的波形图理论推导了断续和连续两种导通模式、多种调制脉宽调制方式、电压和电流两种环路控制方式以及原边反馈和副边反馈两种反馈方式的实现原理。 

第三章首先给出了本论文设计的变换器芯片外部电路结构图,对系统架构中的具体特点、设计指标和内部具体电路结构进行了阐述;
然后对系统的几种损耗进行了介绍,并总结了峰值电流和开关频率对变换器损耗的巨大影响;
针对各种损耗,阐述了变换器芯片的多种模式切换策略,在空载时通过突发模式降低待机功耗、轻载时使用RVS工作模式降低开关损耗、重载时切换到CRM工作模式减小导通损耗;
紧接着对恒流和恒压两种控制模式的原理以及实现方式进行了分析;
最后为了在除了工作模式的策略外更进一步的降低变换器系统的电路损耗并提高转换效率,新颖的提出了退磁时间动态校准和精确谷底导通两种关键技术,分别实现了变压器原副边能量的最佳传递和全负载范围下系统的零电压导通,极大的减小了导通损耗和开关损耗,充分发挥了非对称半桥反激式拓扑在开关电源领域的各种优势。

第四章分别对变换器芯片中欠压锁定电路、内部供电电路、带宽输出范围补偿的峰值电流控制电路、精确谷底导通电路、退磁时间动态校准电路、逻辑控制电路以及各种保护电路等重要模块电路结构进行分析和设计,并验证了其基本功能;
最后针对整个芯片的整体电路进行了各种仿真分析,完成了恒流控制和恒压控制功能,并和第三章提出的具体指标进行对比,验证了该芯片设计的可行性。 

第五章首先对模拟版图具体的设计规则进行了阐述,然后介绍了该变换器芯片的版图设计流程,包括整体的版图布局以及各个电路模块的版图介绍;最后给出整体电路的完整版图,完成了芯片版图设计。 

第六章对本文完成的工作和所获得的成果进行总结,分析设计中考虑不足的问题 并对后续工作进行展望。














