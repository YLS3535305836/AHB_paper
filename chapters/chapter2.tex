\chapter{绪论}

\section{研究背景及意义}

开关电源是一种通过控制晶体管实现控制储能元件进行周期性充电和放电,从而形成稳定输出的电源,随着输出功率的增加,与线性电源相比,开关电源拥有更加优异的工作效率,因此,随着用电设备对转换效率要求越来越高,开关电源已逐步取代传统线性电源。开关电源凭借体积小、重量轻、效率高、可靠性高等优点,广泛应用在工农业生产、交通运输、航空航天、军事以及消费电子等各个领域。开关电源应用于国民生活的方方面面。

近些年来,全球各国政府相继出台有关法律与法规从而制约电源生产厂商提高电源产品的能量转换效率以及降低待机功耗。随着各种法律与法规相继出台,也同时为学术界研究电源技术提供重要方向,促进相应学科蓬勃发展。2013年美国能源部颁布目前仍适用的能效标准DoE VI,此标准主要是针对电源适配器、开关电源以及充电器等电子产品。对于欧盟方面,2016年1月1日正式实行的欧盟能效标准与美国的能效标准类似,相对比较而言,欧盟能效标准对于转换效率性能提出更高要求。

开关电源主要能够分为AC-DC和DC-DC两类,DC-DC开关电源变换器对控制芯片的制造工艺要求较低,常规控制芯片的设计方案以及生产工艺已经很成熟并且有具体标准约束,多种模块化产品得到广泛用户群体的认可。AC-DC变换器由于需要处理高压问题,对工艺要求较高,使得在控制芯片设计与制造过程中存在许多难点,产品可靠性以及效率难以得到提升。AC-DC开关电源作为适配器在消费电子领域已经得到广泛应用,根据不同工况条件,将市电电压转换为直流电压为便携式电子产品提供高质量、稳定的供电,因此市场需求极高。高开关频率、高功率密度、高转换效率以及智能化是AC-DC变换器的主要发展方向。随着人们对能源需求进一步提升,第三代半导体功率器件已经在变换器中得到广泛应用,除了能够降低功率器件的损耗,与此同时也能提高变换器的开关频率,有利于进一步提升变换器的功率密度。我国目前已在半导体功率器件领域取得不菲成绩,但是高端AC-DC控制芯片仍然被国外巨头所垄断,如英飞凌半导体、德州仪器、安森美半导体等。针对目前国内外市场对AC-DC开关电源系统效率和适配器体积进一步提高要求,如何突破工艺限制,降低芯片设计开发成本,缩减设计开发时间,精简电路设计流程,提高控制电路稳定性与可靠性,紧密跟随电源市场需求并实现量产,提高中等功率开关电源控制芯片竞争力,是未来电源行业的重要挑战,同时也是我国电力电子技术进一步发展的关键机遇。突破高性能AC-DC变换器的关键控制技术是我国实现国产芯片全面替代不可或缺的一环,随着我国对集成电路产业持续大力扶持,推动开关电源控制芯片国产化替代并实现性能提升迫在眉睫。

针对目前对便携式电子产品功率、便携性以及高可靠性等需求日益提高,非对称半桥反激式变换器得到了广泛关注。与传统的反激式变换器相比,非对称半桥反激式变换器能够利用原边侧的电流续流特性,在不同工况下功率管都能够很容易实现零电压开启,大大降低开关损耗并减少功率器件热损耗,降低变压器的电压应力,在提高变换器整体转换效率的同时也提升变换器的可靠性。然而,由于非对称半桥结构同时存在多种谐振特性,实现高频、高效率以及高功率密度等特性仍然存在较大难点,对该拓扑结构的最优化控制仍未明确。此外,适配器低成本、绿色化也是其重要发展方向。集成电路控制芯片的裸芯面积直接决定适配器成本,采用更少裸芯面积实现低成本高效控制方案逐渐成为国内外学者的研究热点。针对目前消费电子受众群体之多,设备基数较大的问题,为了达成碳中和目标,优化控制芯片的调控方案降低适配器整体功率损耗,有利于推进绿色能源低碳转型。

目前全球市场对高性能中等输出功率AC-DC开关电源的需求日益增加,基于中等功率开关电源面临着许多待突破的关键技术壁垒,从最优化控制技术的角度提升中等功率开关电源的各项性能,实现开关电源控制芯片国产化替代,在电气化时代迎合绿色低碳环保理念,研究高性能的中等输出功率AC-DC开关电源具有重要意义。



\section{ACDC国内外研究进展}

本节将结合国内外相关参考文献,针对上述提到的非对称半桥反激式变换器高频化、高功率密度、高效率以及智能化等方面,对国内外包括但不限于非对称半桥反激式变换器控制方案研究进行详细调研分析总结。

不对称半桥反激变换器结合了不对称半桥变换器和反激变换器的优点,该变换 器利用变压器漏感实现了软开通,大大降低了变换器的开关损耗,引起了业界广泛 的关注。国内外众多学者对不对称半桥反激变换器进行了深入研究: Seo.D.H 等在 2000 年提出了不对称 PWM 反激拓扑结构,并通过样机验证了拓 扑的可行性;T.-M.Chen 在此基础上,分析了不对称半桥反激变换器用占空比小于 50\%的开关管实现 ZVS 的条件,建立了相应的数学模型,并基于该数学模型设计了 一台输入电压为 400V 负载为 5V/20A 的样机,其效率可达 80\%;Xinyu Xu 通过开 关管实现 ZVS 和输出二极管实现 ZCS 的方法,实现了不对称半桥反激变换器损耗最小化; Bor-Ren Lin 将同步整流技术应用到不对称半桥反激变换器的输出端,并 通过实验证明了该技术能够有效的提高效率;Junseok Cho 等将具有同步整流技术 的不对称半桥反激变换器应用于喷墨打印机,所设计的样机(12V/20A)的效率高达 91\%;T.-M.Chen 提出了对该变换器进行小信号建模,并分析了储能元件和峰值电 流模式控制对传输特性的影响,指出功率级是由两个低频极点和两个高频极点组成 的四阶系统,应用峰值电流模式将传递函数降低到三个极点(一个由滤波电容和负载 确定的主极点,两个由谐振电感和隔直电容确定的高频极点);Li-Ming Wu 和 Chen-Yin Pong 认为不对称半桥反激变换器中的谐振电感和隔直电容是通过谐振传递 能量,是能量传递的部件,而不仅仅是进行线性充放电的器件;Jee-Hoon Jung 和 Joong-Gi Kwon 设计了限制不对称半桥反激变换器的软开关,并分析了最佳谐振条 件,使其能够在大电流场合拥有很高的率效;G. - Y. J e o n g 在不对称半桥反激变换器 中采用了自驱动电压型的同步整流方法,并对其进行了分析;Han Li 对不对称 半桥反激变换器的高频工作状态进行了分析,得到了变换器的稳态电压传输方程和 电压传输函数,并证明了通过电压传递函数设计的变换器比通过线性模型设计的变 换器更易于控制;Sichirollo F.等将不对称半桥反激变换器应用到高亮发光二极管 (HLLED)镇流器中,其开关频率高达 450kHz。

不对称半桥反激变换器具有结构简单,效率高、成本低等优点,非常适合应用 于笔记本电脑的适配器、通信电源和打印机等中小型家用电器的电源,是一种非常 具有研究价值和应用前景的电路拓扑。

\section{论文结构安排}


论文章节安排如下: 

第一章分析反激式AC/DC变换器的原理和发展趋势,阐述研究背景和意义,分析比较国内外电源产业研究情况,对论文的章节内容进行安排。 

第二章分析反激式变换器的拓扑结构和工作原理,结合关键信号的波形图理论推导断续和连续两种工作模式、原边反馈和副边反馈两种反馈方式以及PWM和PFM两种调制方式的实现原理。 

第三章给出本论文设计的芯片整体结构图、重要设计指标、工作方式以及主要特 点,对芯片 PFM 调制下的恒流控制模式和恒压控制模式的实现方式进行阐述,恒压 控制采用重载情况下 PFM 和准谐振共同控制,轻载情况采用 PFM 控制;恒流控制下 采用固定原边电流的峰值以及保持工作周期与退磁时长比值固定的方式实现输出电 流恒定,最后分析减轻开关损耗的谷底导通模式、降低待机功耗的轻载低功耗模式和 优化线性调整率的线电压补偿模式的工作原理,为后续电路设计提供理论支持。 

第四章对电源芯片中启动电路、带隙基准电路、谷底检测电路、原边电流控制电 路、恒流恒压控制等重要模块电路结构进行设计,通过仿真验证电路的功能并基于仿 真结果对电路性能进行分析和优化。 

第五章通过搭建反激式变换器的外围电路,在此基础上对芯片整体电路进行仿真、 完成恒流控制功能、恒压控制功能等性能验证以及电路调整率的仿真和计算,最后完 成芯片版图设计。 

第六章对本文完成的工作和所获得的成果进行总结,分析设计中考虑不足的问题 并对后续工作进行展望。














