\chapter{版图与后仿}

\section{版图设计规则}
%本文采用 BCD 工艺,B、C、D 分别指 bipolar 双极型晶体管、CMOS 场效应晶 体管、DMOS 双扩散场效应晶体管。BCD 工艺内含高低压 MOS 管、LDMOS、NPN 和 PNP 管、肖特基二极管、金属电阻、多晶电阻等器件,同时包含双极型器件和场 效应管的优点,将高压器件低压器件整合到一起,方便了设计者的使用,其低能耗高 效率、高强度、高耐压、开关高速化等特点使得 BCD 工艺成为目前集成电路设计的 常用工艺。由于大部分集成器件在实际生产中是存在偏差和失配的,成因来自于工艺 尺寸、掺杂等器件参数变化、接触电阻、应力梯度等许多因素,通过在设计中遵守版 图设计的匹配规则、合理规划版图设计可以有效提高电路的精度和性能。 提高器件的匹配性是版图设计的一个重点。电阻常因材料温度系数引起的热变化 和热电效应的影响出现误差。多电阻的匹配一般通过共质心版图技术减小热变化影响, 每个电阻拆成偶数个分段阵列并采用叉指结构摆放,每段电阻应避免长度过短而引入 过大的接触电阻误差以及宽度过窄引起的工艺误差。热电效应是由于电阻接触孔处不 同物质的接触会产生接触电势差,其受到温度变化的影响较大。将电阻叉指结构处理, 电阻段一半沿一个方向连接,另一半沿另一方向连接可消除分段热电势。电阻接触孔 分布的位置应靠近。除此之外匹配度要求高的电阻两端加入 dummy 器件可提高匹配 性,与电阻无关的导线避免从匹配电阻上跨过以减小失配等。 MOS 管是模拟电路设计中常见的高匹配度要求的器件。一般采用共质心的版图 布局,将 MOS 管分成多段,需要匹配的器件叉指并行摆放。除此之外保证 MOS 管 摆放方向一致、尽量紧凑匹配 MOS 管的版图、加设 dummy 管、匹配度要求高的晶 体管远离功率模块等方法可以提高 MOS 管的匹配性。 版图中走线规划对版图优化效果也很明显。如数字模块和模拟模块的地线需要分 开防止串扰。布局中避免时钟线和信号线交叠。对于容易受到干扰的重要信号线加设 隔离保护并远离噪声模块。走线使用高层金属可承受更大电流。地线和电源线上均匀 而大量地打孔用以保证良好接触。走线不宜过长避免引起天线效应。除此之外,对版 图模块合理规划,数字模拟模块之间加入隔离,匹配精度要求高的模块远离功率大的 热源模块等方式都可以提高版图的可靠性[40]。 本芯片采用 Nuvoton 0.35μm 高压工艺完成整体电路版图如图 5.12 所示。版图面 积为 920μm×860μm,包含六个 PAD:VDD、GND、CTRL、FB、GATE 和 CS,版图 通过了 DRC、LVS 验证。

版图设计规则是在集成电路(IC)设计过程中用来指导版图设计的一系列规定和约束。它们旨在确保最终的版图设计满足性能、可靠性和制造要求。这些设计规则将负责电路设计的工程师和负责工艺的工程师联系了起来。制定这些版图设计规则是为了将设计的版图进行标准化,在保证电路可靠性的基础上,利用设计规则尽可能减小芯片版图的面积。版图设计规则的重要性在于确保集成电路(IC)设计在实际制造过程中能够被准确重现。确保设计的顺利实施、产品的高性能和可靠性,以及设计成本的控制。通过遵守规则,设计团队可以更好地实现设计目标,提高产品质量[43],推动IC设计的进步和创新。在实际的IC设计中,不同的工艺以及不同的厂商使用的设计规则往往不同,在版图绘制过程中,常用的设计规则主要包括:

1、线宽和间距规则:规定了不同金属层或多晶硅层中导线的最小宽度和最小间距。这些规则通常由工艺技术能够实现的最小特征尺寸和工艺容忍度所决定。

2、接触孔规则:规定了不同层之间的接触孔的最小尺寸和间距,以确保层与层之间的连接正常进行。

3、阱孔规则:规定了用于隔离衬底和金属层之间的阱孔的尺寸和间距。

4、悬空金属规则:规定了悬空金属结构的最小尺寸和支撑要求,以确保金属线的稳定性和可靠性。

5、器件尺寸和布局规则:规定了器件的最小尺寸、引脚位置和布局要求,以确保器件功能和性能的正常表现。

6、电源线规则:规定了电源线的宽度和间距,以确保提供足够的电流和降低电阻。

7、设计规则检查(DRC)规则:规定了版图设计需要满足的几何和拓扑约束,以便进行自动化的设计规则检查。

8、特殊工艺规则:针对特殊工艺步骤(如电池区域、ESD保护区域等)制定的特殊版图设计规则。

版图设计规则的重要性在于确保集成电路(IC)设计在实际制造过程中能够被准确重现。确保设计的顺利实施、产品的高性能和可靠性,以及设计成本的控制。通过遵守规则,设计师可以更好地实现设计目标,提高产品质量[44],推动IC设计的进步和创新。

\section{版图设计流程}

\section{具体版图设计}

根据版图的设计流程,首先将模块进行划分,然后进行布局连线,本文设计的非对称半桥反激式变换器芯片主要有带隙基准模块、逻辑控制模块、输出驱动模块等构成。完整的版图包括数字模块和模拟模块以及存储一些数字逻辑信号的存储器模块,由于本文主要设计的是模拟部分,所以只展示模拟模块的版图。

\subsection{版图布局}

首先对整个变换器进行版图布局,为了减小数字信号逻辑切换产生的衬底耦合问题,将数字模块和模拟模块进行分开布局,对于模拟部分,本文设计了16个输出通道,为了保证通道之间相互匹配,将16个输出通道对称放在左边,其余模块放在右边,下方为数字部分的版图,因此本文设计的恒流驱动芯片总体布局如图所示。

\subsection{各个模块版图}

\subsection{整体版图}


\section{小结}




